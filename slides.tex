\documentclass[notes]{beamer}
\mode<presentation>
{
  \usetheme{Luebeck}
  \useoutertheme[subsection=false,footline=authortitle]{miniframes}
  \usecolortheme{dolphin}
  \usefonttheme{default}
  \setbeamertemplate{navigation symbols}{}
  \setbeamertemplate{caption}[numbered]
} 

\usepackage[english]{babel}
\usepackage[utf8]{inputenc}
\usepackage{mathtools}
\usepackage{graphicx}
\graphicspath{ {graphics/} }

\title[Master's Thesis Presentation]{Can web pages be distinguished by their side channel signatures?}
\author{Emil 'Skeen' Madsen - 20105376}
\institute{Aarhus University - Computer Science}
\date{}

\begin{document}

\begin{frame}
  \titlepage
  \includegraphics[scale=0.5]{logo}
\end{frame}

\section*{Introduction}
\begin{frame}{Introduction}
\tableofcontents
\end{frame}

\section[Paper]{Paper - Website Fingerprinting at Internet Scale}
\begin{frame}{Presentation of paper}
\begin{itemize}
\item What is the idea
\item What does it do
\item How does it do it
\end{itemize}
\end{frame}

\note{For the presentation, you will have 30 min. I suggest you spend about 15-min on the presentation on the paper and what it does and then put its results in the context of your project – some ideas from these papers are not applicable (they all focus on network-based traffic analysis) – but what can one learn from them w.r.t. browser-based traffic analysis that you performed? Are the classification techniques or defense mechanisms discussed defenses applicable; how does the evaluation
methodology and results compare to yours, what did you learn from the related work that these papers describe, etc. [full disclosure: I have a cursory familiarity with these papers so I’m looking forward to learning something 😃]}

\section[Project]{Paper versus Project}
\begin{frame}{Project}
\begin{itemize}
\item Item one
\end{itemize}
\end{frame}

\section{Missingno}
\begin{frame}{Missing}
\begin{itemize}
\item Item one
\end{itemize}
\end{frame}

\end{document}

